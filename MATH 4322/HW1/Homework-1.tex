% Options for packages loaded elsewhere
\PassOptionsToPackage{unicode}{hyperref}
\PassOptionsToPackage{hyphens}{url}
%
\documentclass[
]{article}
\usepackage{amsmath,amssymb}
\usepackage{lmodern}
\usepackage{iftex}
\ifPDFTeX
  \usepackage[T1]{fontenc}
  \usepackage[utf8]{inputenc}
  \usepackage{textcomp} % provide euro and other symbols
\else % if luatex or xetex
  \usepackage{unicode-math}
  \defaultfontfeatures{Scale=MatchLowercase}
  \defaultfontfeatures[\rmfamily]{Ligatures=TeX,Scale=1}
\fi
% Use upquote if available, for straight quotes in verbatim environments
\IfFileExists{upquote.sty}{\usepackage{upquote}}{}
\IfFileExists{microtype.sty}{% use microtype if available
  \usepackage[]{microtype}
  \UseMicrotypeSet[protrusion]{basicmath} % disable protrusion for tt fonts
}{}
\makeatletter
\@ifundefined{KOMAClassName}{% if non-KOMA class
  \IfFileExists{parskip.sty}{%
    \usepackage{parskip}
  }{% else
    \setlength{\parindent}{0pt}
    \setlength{\parskip}{6pt plus 2pt minus 1pt}}
}{% if KOMA class
  \KOMAoptions{parskip=half}}
\makeatother
\usepackage{xcolor}
\usepackage[margin=1in]{geometry}
\usepackage{color}
\usepackage{fancyvrb}
\newcommand{\VerbBar}{|}
\newcommand{\VERB}{\Verb[commandchars=\\\{\}]}
\DefineVerbatimEnvironment{Highlighting}{Verbatim}{commandchars=\\\{\}}
% Add ',fontsize=\small' for more characters per line
\usepackage{framed}
\definecolor{shadecolor}{RGB}{248,248,248}
\newenvironment{Shaded}{\begin{snugshade}}{\end{snugshade}}
\newcommand{\AlertTok}[1]{\textcolor[rgb]{0.94,0.16,0.16}{#1}}
\newcommand{\AnnotationTok}[1]{\textcolor[rgb]{0.56,0.35,0.01}{\textbf{\textit{#1}}}}
\newcommand{\AttributeTok}[1]{\textcolor[rgb]{0.77,0.63,0.00}{#1}}
\newcommand{\BaseNTok}[1]{\textcolor[rgb]{0.00,0.00,0.81}{#1}}
\newcommand{\BuiltInTok}[1]{#1}
\newcommand{\CharTok}[1]{\textcolor[rgb]{0.31,0.60,0.02}{#1}}
\newcommand{\CommentTok}[1]{\textcolor[rgb]{0.56,0.35,0.01}{\textit{#1}}}
\newcommand{\CommentVarTok}[1]{\textcolor[rgb]{0.56,0.35,0.01}{\textbf{\textit{#1}}}}
\newcommand{\ConstantTok}[1]{\textcolor[rgb]{0.00,0.00,0.00}{#1}}
\newcommand{\ControlFlowTok}[1]{\textcolor[rgb]{0.13,0.29,0.53}{\textbf{#1}}}
\newcommand{\DataTypeTok}[1]{\textcolor[rgb]{0.13,0.29,0.53}{#1}}
\newcommand{\DecValTok}[1]{\textcolor[rgb]{0.00,0.00,0.81}{#1}}
\newcommand{\DocumentationTok}[1]{\textcolor[rgb]{0.56,0.35,0.01}{\textbf{\textit{#1}}}}
\newcommand{\ErrorTok}[1]{\textcolor[rgb]{0.64,0.00,0.00}{\textbf{#1}}}
\newcommand{\ExtensionTok}[1]{#1}
\newcommand{\FloatTok}[1]{\textcolor[rgb]{0.00,0.00,0.81}{#1}}
\newcommand{\FunctionTok}[1]{\textcolor[rgb]{0.00,0.00,0.00}{#1}}
\newcommand{\ImportTok}[1]{#1}
\newcommand{\InformationTok}[1]{\textcolor[rgb]{0.56,0.35,0.01}{\textbf{\textit{#1}}}}
\newcommand{\KeywordTok}[1]{\textcolor[rgb]{0.13,0.29,0.53}{\textbf{#1}}}
\newcommand{\NormalTok}[1]{#1}
\newcommand{\OperatorTok}[1]{\textcolor[rgb]{0.81,0.36,0.00}{\textbf{#1}}}
\newcommand{\OtherTok}[1]{\textcolor[rgb]{0.56,0.35,0.01}{#1}}
\newcommand{\PreprocessorTok}[1]{\textcolor[rgb]{0.56,0.35,0.01}{\textit{#1}}}
\newcommand{\RegionMarkerTok}[1]{#1}
\newcommand{\SpecialCharTok}[1]{\textcolor[rgb]{0.00,0.00,0.00}{#1}}
\newcommand{\SpecialStringTok}[1]{\textcolor[rgb]{0.31,0.60,0.02}{#1}}
\newcommand{\StringTok}[1]{\textcolor[rgb]{0.31,0.60,0.02}{#1}}
\newcommand{\VariableTok}[1]{\textcolor[rgb]{0.00,0.00,0.00}{#1}}
\newcommand{\VerbatimStringTok}[1]{\textcolor[rgb]{0.31,0.60,0.02}{#1}}
\newcommand{\WarningTok}[1]{\textcolor[rgb]{0.56,0.35,0.01}{\textbf{\textit{#1}}}}
\usepackage{graphicx}
\makeatletter
\def\maxwidth{\ifdim\Gin@nat@width>\linewidth\linewidth\else\Gin@nat@width\fi}
\def\maxheight{\ifdim\Gin@nat@height>\textheight\textheight\else\Gin@nat@height\fi}
\makeatother
% Scale images if necessary, so that they will not overflow the page
% margins by default, and it is still possible to overwrite the defaults
% using explicit options in \includegraphics[width, height, ...]{}
\setkeys{Gin}{width=\maxwidth,height=\maxheight,keepaspectratio}
% Set default figure placement to htbp
\makeatletter
\def\fps@figure{htbp}
\makeatother
\setlength{\emergencystretch}{3em} % prevent overfull lines
\providecommand{\tightlist}{%
  \setlength{\itemsep}{0pt}\setlength{\parskip}{0pt}}
\setcounter{secnumdepth}{-\maxdimen} % remove section numbering
\usepackage{amsbsy}
\ifLuaTeX
  \usepackage{selnolig}  % disable illegal ligatures
\fi
\IfFileExists{bookmark.sty}{\usepackage{bookmark}}{\usepackage{hyperref}}
\IfFileExists{xurl.sty}{\usepackage{xurl}}{} % add URL line breaks if available
\urlstyle{same} % disable monospaced font for URLs
\hypersetup{
  pdftitle={MATH 4322 Homework 1},
  pdfauthor={Saim Ali},
  hidelinks,
  pdfcreator={LaTeX via pandoc}}

\title{MATH 4322 Homework 1}
\author{Saim Ali}
\date{1/28/23}

\begin{document}
\maketitle

\hypertarget{instructions}{%
\subsection{Instructions}\label{instructions}}

\begin{enumerate}
\def\labelenumi{\arabic{enumi}.}
\tightlist
\item
  Due date: January 31, 2023, 11:59 PM
\item
  Answer the questions fully for full credit.
\item
  Scan or Type your answers and submit only one file. (If you submit
  several files only the recent one uploaded will be graded).
\item
  Preferably save your file as PDF before uploading.
\item
  Submit in Canvas under Homework 1.
\item
  These questions are from \emph{An Introduction to Statistical
  Learning}, second edition by James, et. al., chapter 2.
\item
  The information in the gray boxes are R code that you can use to
  answer the questions.
\end{enumerate}

\hypertarget{problem-1}{%
\subsection{Problem 1}\label{problem-1}}

Explain whether each scenario is a classification or regression problem,
and indicate whether we are most interested in inference or prediction.
Finally, provide \(n\) and \(p\).

\begin{enumerate}
\def\labelenumi{\alph{enumi})}
\tightlist
\item
  We are interested in predicting the \% change in the USD/Euro exchange
  rate in relation to the weekly changes in the world stock markets.
  Hence we collect weekly data for all of 2012. For each week we record
  the \% change in the USD/Euro, the \% change in the US market, the \%
  change in the British market, and the \% change in the German
  market.\\
  \textbf{hello }
\item
  An online store is determining whether or not a customer will purchase
  additional items. This online store collected data from 1500 customers
  and looked at cost of initial purchase, if there was a special offer,
  type of item purchased, number of times the customer logged into their
  account, and if they purchased additional items.
\end{enumerate}

\hypertarget{problem-2}{%
\subsection{Problem 2}\label{problem-2}}

This is an exercises about bias, variance and MSE.

Suppose we have \(n\) independent Bernoulli trails with true success
probability \(p\). Consider two estimators of \(p\):
\(\hat{p}_1 = \hat{p}\) where \(\hat{p}\) is the sample proportion of
successes and \(\hat{p}_2 = \frac{1}{2}\), a fixed constant.

\begin{enumerate}
\def\labelenumi{\alph{enumi})}
\tightlist
\item
  Find the expected value and bias of each estimator.\\
\item
  Find the variance of each estimator.\\
\item
  Find the MSE of each estimator and compare them by plotting against
  the true \(p\). Use \(n = 4\). Comment on the comparison.
\end{enumerate}

\newpage

\hypertarget{problem-3}{%
\subsection{Problem 3}\label{problem-3}}

Describe the differences between a parametric and a non-parametric
statistical learning approach. What are the advantages of a parametric
approach to regression or classification (as opposed to a non-parametric
approach)? What are its disadvantages?

\hypertarget{problem-4}{%
\subsection{Problem 4}\label{problem-4}}

This exercise involves the \texttt{Auto} data set in \texttt{ISLR}
package. Make sure that the missing values have been removed from the
data.

\begin{enumerate}
\def\labelenumi{(\alph{enumi})}
\tightlist
\item
  Which of the predictors are quantitative, and which are qualitative?\\
\item
  What is the range of each quantitative predictor? You can answer this
  using the \texttt{summary()} function.\\
\item
  What is the mean and standard deviation of each quantitative
  predictor?\\
\item
  Now remove the 10th through 85th observations. What is the range,
  mean, and standard deviation of each predictor in the subset of the
  data that remains?\\
\item
  Using the full data set, investigate the predictors graphically, using
  scatterplots or other tools of your choice. Create some plots
  highlighting the relationships among the predictors. Comment on your
  findings.\\
\item
  Suppose that we wish to predict gas mileage (mpg) on the basis of the
  other variables. Do your plots suggest that any of the other variables
  might be useful in predicting mpg? Justify your answer.
\end{enumerate}

\hypertarget{problem-5}{%
\subsection{Problem 5}\label{problem-5}}

This exercise relates to the \texttt{College} data set, which can be
found in the file \texttt{College.csv} attached to this homework set in
Blackboard. It contains a number of variables for 777 different
universities and colleges in the US. The variables are

\begin{itemize}
\tightlist
\item
  Private : Public/private indicator
\item
  Apps : Number of applications received
\item
  Accept : Number of applicants accepted
\item
  Enroll : Number of new students enrolled
\item
  Top10perc : New students from top 10\% of high school class
\item
  Top25perc : New students from top 25\% of high school class
\item
  F.Undergrad : Number of full-time undergraduates
\item
  P.Undergrad : Number of part-time undergraduates
\item
  Outstate : Out-of-state tuition
\item
  Room.Board : Room and board costs
\item
  Books : Estimated book costs
\item
  Personal : Estimated personal spending
\item
  PhD : Percent of faculty with Ph.D.'s
\item
  Terminal : Percent of faculty with terminal degree
\item
  S.F.Ratio : Student/faculty ratio
\item
  perc.alumni : Percent of alumni who donate
\item
  Expend : Instructional expenditure per student
\item
  Grad.Rate : Graduation rate
\end{itemize}

Before reading the data into \texttt{R}, it can be viewed in Excel or a
text editor.

\begin{enumerate}
\def\labelenumi{\alph{enumi})}
\item
  Use the \texttt{read.csv()} function to read the data into \texttt{R}.
  Call the loaded data \texttt{college}. Make sure that you have the
  directory set to the correct location for the data. You can also
  import this data set into \texttt{RStudio} by using the \textbf{Import
  Dataset} \(\rightarrow\) \textbf{From Text} drop down list in the
  Environment window.
\item
  Look at the data using the \texttt{View()} function. You should notice
  that the first column is just the name of each university. We will not
  use this column as a variable but it may be handy to have these names
  for later. Try the following commands in \texttt{R}:
\end{enumerate}

\begin{Shaded}
\begin{Highlighting}[]
\FunctionTok{rownames}\NormalTok{(college) }\OtherTok{\textless{}{-}}\NormalTok{ college[,}\DecValTok{1}\NormalTok{]}
\NormalTok{college }\OtherTok{\textless{}{-}}\NormalTok{ college[,}\SpecialCharTok{{-}}\DecValTok{1}\NormalTok{]}
\FunctionTok{View}\NormalTok{(college)}
\end{Highlighting}
\end{Shaded}

If you are getting an error make sure your data frame is named with a
lowercase ``c''.\\
Give a brief description of what you see in the data frame.

\begin{enumerate}
\def\labelenumi{\alph{enumi})}
\setcounter{enumi}{2}
\tightlist
\item
  Use the \texttt{summary()} function to produce a numerical summary of
  the variables in the data set. Is there any variables that do not show
  a numerical summary?
\end{enumerate}

Type in the following in \texttt{R}:

\begin{Shaded}
\begin{Highlighting}[]
\NormalTok{college}\SpecialCharTok{$}\NormalTok{Private }\OtherTok{\textless{}{-}} \FunctionTok{as.factor}\NormalTok{(college}\SpecialCharTok{$}\NormalTok{Private)}
\end{Highlighting}
\end{Shaded}

\begin{enumerate}
\def\labelenumi{\alph{enumi})}
\setcounter{enumi}{3}
\tightlist
\item
  Use the \texttt{pairs()} function to produce a scatterplot matrix of
  the first five columns or variable of the dataset. Describe any
  relationships you see in these plots.\\
\item
  Use the \texttt{plot()} function to produce a plot of
  \texttt{Outstate} versus \texttt{Private}. What type of plot was
  produced? Give a description of the relationship. \emph{Hint:
  `Outstate is in the y-axis}.\\
\item
  Create a new qualitative variable, called \texttt{Elite}, by
  \(binning\) the \texttt{Top10perc} variable. We are going to divide
  universities into two groups based on whether or not the proportion of
  students coming from the top 10\% of their high school classes exceeds
  50\%. Type in the following in \texttt{R}:
\end{enumerate}

\begin{Shaded}
\begin{Highlighting}[]
\NormalTok{Elite }\OtherTok{\textless{}{-}} \FunctionTok{rep}\NormalTok{(}\StringTok{"No"}\NormalTok{, }\FunctionTok{nrow}\NormalTok{(college)) }\CommentTok{\#this gives a column of No\textquotesingle{}s for the same number of rows college. }
\NormalTok{Elite[college}\SpecialCharTok{$}\NormalTok{Top10perc }\SpecialCharTok{\textgreater{}} \DecValTok{50}\NormalTok{] }\OtherTok{\textless{}{-}} \StringTok{"Yes"} \CommentTok{\#changes to Yes if top 10\% is greater than 50}
\NormalTok{Elite }\OtherTok{\textless{}{-}} \FunctionTok{as.factor}\NormalTok{(Elite)}
\NormalTok{college }\OtherTok{\textless{}{-}} \FunctionTok{data.frame}\NormalTok{(college,Elite) }\CommentTok{\#adds Elite as a column}
\end{Highlighting}
\end{Shaded}

Use the \texttt{summary()} function to see how many elite universities
there are.

\hypertarget{problem-6}{%
\subsection{Problem 6}\label{problem-6}}

This exercise involves the Boston housing data set.

\begin{enumerate}
\def\labelenumi{(\alph{enumi})}
\tightlist
\item
  To begin, load in the Boston data set. The Boston data set is part of
  the ISLR2 library. You may have to install the ISLR2 library then call
  for this library.
\end{enumerate}

\begin{Shaded}
\begin{Highlighting}[]
\FunctionTok{library}\NormalTok{(ISLR2)}
\end{Highlighting}
\end{Shaded}

Now the data set is contained in the object Boston.

\begin{Shaded}
\begin{Highlighting}[]
\NormalTok{Boston}
\end{Highlighting}
\end{Shaded}

Read about the data set:

\begin{Shaded}
\begin{Highlighting}[]
\NormalTok{?Boston}
\end{Highlighting}
\end{Shaded}

How many rows are in this data set? How many columns? What do the rows
and columns represent?

\begin{enumerate}
\def\labelenumi{(\alph{enumi})}
\setcounter{enumi}{1}
\item
  Make some pairwise scatterplots of the predictors (columns) in this
  data set. Describe your findings.
\item
  Are any of the predictors associated with per capita crime rate? If
  so, explain the relationship.
\item
  Do any of the census tracts of Boston appear to have particularly high
  crime rates? Tax rates? Pupil-teacher ratios? Comment on the range of
  each predictor.
\item
  How many of the census tracts in this data set bound the Charles
  river?
\item
  What is the median pupil-teacher ratio among the towns in this data
  set?
\item
  Which census tract of Boston has lowest median value of owner occupied
  homes? What are the values of the other predictors for that census
  tract, and how do those values compare to the overall ranges for those
  predictors? Comment on your findings.
\item
  In this data set, how many of the census tracts average more than
  seven rooms per dwelling? More than eight rooms per dwelling? Comment
  on the census tracts that average more than eight rooms per dwelling.
\end{enumerate}

\end{document}
